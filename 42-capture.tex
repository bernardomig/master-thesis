\section{Capture}
\label{section:capture}

As seen before, acquisitions only capture a subset of the scene geometry and color, so multiple acquisitions are required. This problem can be partially solved by recording multiple acquisitions instead of one. Therefore, a capture is a collection of acquisitions of the same scene and its goal is to collect enough data to create a fully 3D reconstruction. However, this raises some challenges, on how to plan and execute the multitude of acquisitions and how to merge the data from all of the acquisitions (discussed in \cref{section:acquisition-registration}).

Planning determines where should the 3D scanner be placed in each acquisition and the sequence of the acquisitions. In this work this was done with the objective to maintain a minimum point density on all surfaces, capture color information of as much surfaces as possible and minimize the processing errors. Each one of this problems and its solutions are explained in more detail hereupon.

To begin with, occlusion and range limitations restrict the covered area of an acquisition to a subset of the scene, which is dependent of the position and orientation of the 3D scanner in the scene.

Secondly, the point density decreases with the distance of the object to the sensor, which can influence the reconstruction, specially if small objects exist. For example, a wall does not need a high point density, but a smaller object such as a chair or table should have a higher one. Therefore, the position and orientation of the acquisitions should regard this, such that the point density is adequate to the dimensions of the objects.

At last, the acquisition registration requires that between each acquisition there is enough overlap between the point clouds, so enough correspondent points exists to compute the registration between acquisitions. So, between each acquisition there should be a maximum distance, such that this registration is possible. Also, this registration requires a good initial estimate for the transformation, otherwise it is not able to find a correct transformation. The solution proposed is to define a sequence of acquisitions such that each subsequent acquisition is near to the previous one and the relative rotation is small.

In conclusion, a good capture planning requires that key acquisitions are made to minimize occlusion and maintain a adequate point density and multiple acquisition have to be made, connecting the key points, and each acquisitions should be close enough to the previous one, such that the registration between acquisitions are possible. In this work, we determined this sequence of acquisitions by determining a path inside the scene. This process, however, can be very subjective and dependent of the user, and the evaluation of the capture is all done afterward, because no feedback exists during the capture, which is a disadvantage in comparison with other reconstruction systems like the \textit{Google Tango}.

