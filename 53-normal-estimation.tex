\section{Normal Estimation}
\label{section:normal-estimation}

As part of the reconstruction work, it is common to create a mesh using the point cloud, through a process called triangulation. Despite that in this work, this was not performed, it can be done in posterior work. Surface normals are a important property of geometric surfaces and are a requirement for most triangulation algorithms. Also, normals are required for lightning calculation, which can improve the rendering of a point cloud model\footnote{See "Estimating Surface Normals in a PointCloud" in \url{http://pointclouds.org/documentation/tutorials/normal_estimation.php}.}. As an example, the Stanford Bunny model\footnote{From \url{https://www.cc.gatech.edu/~turk/bunny/bunny.html}.} rendered with and without lightning are show in \cref{figure:bunny-normals}. As can be seen, lightning can improve the perception of the geometry of the point cloud model.

Normal estimation is quite trivial for surfaces, but for point clouds the process is quite not as easy. Usually there are two ways to estimate the normals: either by meshing the surface first, and then calculate the normals for the mesh, or using the point cloud itself to infer the normals. However, most meshing algorithms already require the normals to achieve a good result, so the latter option is more effective.

\begin{figure}[h]
    
    \centering
    \begin{subfigure}{0.4\textwidth}
        \centering
        \includegraphics[height=6cm]{bunny-with-normals}     
    \end{subfigure}%
    \begin{subfigure}{0.4\textwidth}
        \centering
        \includegraphics[height=6cm]{bunny-without-normals}     
    \end{subfigure}

    \caption{Stanford rabbit rendering with lightning (on the left), using the normals information, and without lightning (on the right).}
    \label{figure:bunny-normals}

\end{figure}

The most common solution is, for each point, to find the $k$ closest points, defined as the $k$-neighborhood of a point, and calculate the normal of the best-fitting plane formed by this points. However, finding the $k$-neighborhood of all the points in a point cloud has a time complexity $O(N \log N)$, so it can become quite slow for point clouds with a large number of points. In this work, an alternative solution was used to find the closest points, exploiting the bidimensional structure of the point cloud. This solution has a linear time complexity $O(N)$, which makes it a valuable solution for large point clouds.

The solution uses the fact that each point in the point cloud resulting from \cref{section:point-registration} has two indexes, one for the range index $i$ and one for the laser scan $j$. For each laser scan, each point $p_i$ has a neighborhood ${p_{i-k}, \ \dots, \ p_{i+k}}$, because each subsequent point has an increasing angle to the previous point. Between successive laser scans each point has an increasing angle (the pan angle) to the previous one. Therefore, for this algorithm, the neighborhood of each point: 

\begin{equation}
\label{eqn:point-neighborhood}
    neighborhood(p_{i,j}, k_1, k_2) = \{p_{i-k_1, j-k_2}, \ \dots, \ p_{i+k_1, j+k_2}\}.
\end{equation}

\noindent
The value of $k_1$ and $k_2$ have to be adjusted for a better result, because if the values are large, fine details are going to disappear and edges are going to be smeared, and on the other hand if the values are small, the surface will appear as too noisy. In this work, the value of $k_1$ and $k_2$ was 3, so the neighbor has 9 points.

Then, for each point, the tangent plane that fits the neighborhood is calculated, which in turn is a least-square plane fitting problem. This is usually solved by an analysis of Principal Component Analysis, as explained in \cref{section:calibration-cost-function}. This method will compute the direction of the normal $n$ for each point.

Then, the orientation of the point has to be defined, because the result of the PCA is ambiguous, which may lead to inconsistent normals in the point cloud. In this case, the solution found was to orientate the normals towards the frame of the 3D scanner, which for each acquisition in the origin of the coordinate system. Therefore, each normal has to satisfy:

\begin{equation}
\label{eqn:normal-orientation}
    n \cdot p < 0.
\end{equation}
