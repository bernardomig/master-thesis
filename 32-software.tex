\section{Software}
\label{section:software}

This section presents the software developed or used in this work. The code was split into two repositories: the 3D mobile scanner runtime (\cref{section:software_lemonbot_software}) and the data processing algorithms (\cref{section:software_data_processing}).

\subsection{Lemonbot software}
\label{section:software_lemonbot_software}

One of the key components of robots is the software and most robots require large amounts of code to function. In older robot architectures, the code was a monolith responsible for every system of the robot, however this proven to be an inefficient approach, as it required a high understanding of all the code by the developer, created dependencies and collisions between separated parts of the codebase and required a high maintenance. This also created a single point of failure, because if just one component failed everything failed. 

This required a paradigm shift towards distributed systems. In a distributed system, each node runs isolated and interact through message passing. This paradigm can be seen all across computer systems and a great example of this paradigm is the Erlang language runtime system, which was designed to be distributed, fault-tolerant and highly available. This system was used in the production of a high-reliable ATM switch - the Ericsson AXD301 - to achieve an outstanding \SI{99.9999999}{\percent} reliability.

This systems are known to be:

\begin{itemize}
    \item Fault-tolerant. A failure in one node does not affect other nodes, so there is not a overall system crash, unlike monolithic systems. Usually, because errors are transitive and not very frequent, a restart-on-failure policy is used to keep the downtime low.
    \item More generic. Because each node has a single responsibility, they tend to be more generic and detached to a single project, so integration in a new project can be easy. This is fundamental to reduce the need to "reinvent the wheel", therefore reducing the development cost and time of this complex systems.
    \item Easier to develop. Each component can be developed be separate developers, with different languages and independent release cycles, because they do not have any dependency between each over and only a message specification needs to be agreed on between developer teams, making collaborative development possible. Debugging is also easier, because each node can be unit-tested separated with the "real environment".
    \item Decoupled. Each node runs in a isolated environment, to mitigate all the conflicts that can appear in a unknown environment, like resource utilization and shared library versioning. This can go as far as running each node in a separate virtual machine, by using containers.
\end{itemize}

However, this systems now requires an underlying infrastructure, responsible for:

\begin{itemize}
    \item Orquestration. Nodes run inside this distributed environment, so a special orquestration software is responsible to start and keep track of all the running nodes.
    \item Communication. Each node communicates to other components by message passing, with different topologies, like publisher-subscriber or client-server. However, this requires a standardization on the protocol of communication and message serialization.
    \item Discovery. Each components needs to find the location on the network of the other nodes, to be able to communicate to then.
    \item Configuration. Each parameter of the system is saved in a key-value store, accessible to all the nodes.
\end{itemize}

\subsubsection{ROS}

In robotics, such a system is being developed and it's quickly becoming the new standard: the Robot Operating System. ROS is a software architecture for robot development, providing a collection of tools, libraries and conventions to simplify the development of complex robot systems. It was originally created by the Stanford University in the mid-2000s and now is being widely adopted by most research communities.

ROS is split into 3 levels, according to~\cite{fernandez_2015}: the \textit{filesystem} level, the \textit{the computational graph} level and the \textit{community} level. Each level is composed of several core components, that make the whole system work, as seen in \cref{figure:ros_overview}.

\begin{figure}[h]
    \label{figure:ros_overview}
        
        \caption{ROS architecture overview}
    
        \tikzstyle{ros_level}=[rectangle, draw, text centered, thick, text width=8em, rounded corners, font=\bfseries]
        \tikzstyle{level}=[rectangle, draw, font=\bfseries, thick, rounded corners, inner xsep=0.5cm]
        \tikzstyle{component}=[rectangle, draw, text width=9em]
    
    
        \centering
        \begin{tikzpicture}[scale=0.9]
            
            \node[ros_level] (ROS) {ROS};
            \coordinate[below=0.4cm of ROS] (aux_node1);
            \node[level, below=0.6cm of aux_node1] (computational_graph) {Computational Graph};
            \node[level, left=1.7cm of computational_graph] (filesystem) {Filesystem};
            \node[level, right=1cm of computational_graph] (community) {Community};
    
            \draw[->] (ROS) -- (aux_node1) -- (computational_graph.north);
            \draw[->] (ROS) -- (aux_node1) -| (filesystem.north);
            \draw[->] (ROS) -- (aux_node1) -| (community);
    
            \coordinate[below=of filesystem.west, xshift=0.5cm] (aux_node2);
            \coordinate[below=of computational_graph.west, xshift=0.5cm] (aux_node3);
            \coordinate[below=of community.west, xshift=0.5cm] (aux_node4);
    
            \node[component, anchor=west, yshift=0cm] (meta_packages) at (aux_node2) {Meta Packages};
            \node[component, anchor=west, yshift=-0.8cm] (packages) at (aux_node2) {Packages};
            \draw[->] (filesystem.west) -- ++(-0.2cm, 0) |- (meta_packages.west);
            \draw[->] (filesystem.west) -- ++(-0.2cm, 0) |- (packages.west);
    
            \node[component, anchor=west, yshift=0cm] (nodes) at (aux_node3) {Nodes};
            \node[component, anchor=west, yshift=-0.8cm] (master) at (aux_node3) {Master};
            \node[component, anchor=west, yshift=-1.6cm] (param_server) at (aux_node3) {Parameter Server};
            \node[component, anchor=west, yshift=-2.4cm] (messages) at (aux_node3) {Messages};
            \node[component, anchor=west, yshift=-3.2cm] (topics) at (aux_node3) {Topics};
            \node[component, anchor=west, yshift=-4.0cm] (services) at (aux_node3) {Services};
            \node[component, anchor=west, yshift=-4.8cm] (bags) at (aux_node3) {Bags};
            \draw[->] (computational_graph.west) -- ++(-0.2cm, 0) |- (nodes.west);
            \draw[->] (computational_graph.west) -- ++(-0.2cm, 0) |- (master.west);
            \draw[->] (computational_graph.west) -- ++(-0.2cm, 0) |- (param_server.west);
            \draw[->] (computational_graph.west) -- ++(-0.2cm, 0) |- (messages.west);
            \draw[->] (computational_graph.west) -- ++(-0.2cm, 0) |- (topics.west);
            \draw[->] (computational_graph.west) -- ++(-0.2cm, 0) |- (services.west);
            \draw[->] (computational_graph.west) -- ++(-0.2cm, 0) |- (bags.west);
    
            \node[component, anchor=west, yshift=0cm] (distributions) at (aux_node4) {Distributions};
            \node[component, anchor=west, yshift=-0.8cm] (repositories) at (aux_node4) {Repositories};
            \node[component, anchor=west, yshift=-1.6cm] (ros_wiki) at (aux_node4) {ROS Wiki};
            \node[component, anchor=west, yshift=-2.4cm] (bug_ticket_system) at (aux_node4) {Bug Ticket System};
            \node[component, anchor=west, yshift=-3.2cm] (mailing_lists) at (aux_node4) {Mailing Lists};
            \node[component, anchor=west, yshift=-4.0cm] (ros_answers) at (aux_node4) {Ros Answers};
            \node[component, anchor=west, yshift=-4.8cm] (blog) at (aux_node4) {Blog};
            \draw[->] (community.west) -- ++(-0.2cm, 0) |- (distributions.west);
            \draw[->] (community.west) -- ++(-0.2cm, 0) |- (repositories.west);
            \draw[->] (community.west) -- ++(-0.2cm, 0) |- (ros_wiki.west);
            \draw[->] (community.west) -- ++(-0.2cm, 0) |- (bug_ticket_system.west);
            \draw[->] (community.west) -- ++(-0.2cm, 0) |- (mailing_lists.west);
            \draw[->] (community.west) -- ++(-0.2cm, 0) |- (ros_answers.west);
            \draw[->] (community.west) -- ++(-0.2cm, 0) |- (blog.west);
    
        \end{tikzpicture}
    
    \end{figure}

{\huge Maybe add more explanations???
    - messages
}

\subsubsection{Coordinate Frames}

One of the biggest challenges in robot programming is to keep track of all the coordinate frames present in the robot. This is challenging because the measurements of each transformation can be asynchronous and at different scanning rates, but queries are made between any two coordinate frames and at any time. To solve this problem, a special package in ROS - the tf package - stores all the measured transformations as a tree graph (more specifically, a directed acyclic graph), and interpolates between the measured transformations to obtain a approximate transformations at any time. 



\subsection{Data Processing Algorithms}
\label{section:software_data_processing}