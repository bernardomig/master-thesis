\section{Point Registration}
\label{section:point-registration}

Each laser scan is a collection of points in polar coordinates, so each range point $(r_i, \theta_i)$ is transformed to a point in the laser frame of reference according to \cref{eqn:range-to-point}. The angles are uniform distributed between a minimum and maximum angle, $\theta_{min}$ and $\theta_{max}$, respectively, so $\theta_i = \{\theta_{min}, \ \dots, \ \theta_{max}\}, i=1 \dots N$. The index $i$ is defined as the range index of each laser scan.

\begin{equation}\label{eqn:range-to-point}
    \left[
        \begin{array}{c}
            x_i \\
            y_i \\
            z_i \\
        \end{array}
    \right]
    =
    \left[
        \begin{array}{c}
            r_i \cos(\theta_i) \\
            r_i \sin(\theta_i) \\
            0 \\
        \end{array}    
    \right]
\end{equation}

Further on, each point $p_{ij}$ is registered in the referencial of the acquisition. According to the transformation graph (see \cref{figure:geometric-transformation-graph}), there are two transformation from the acquisition frame and the laser scanner frame: the transform from the acquisition frame to the PTU frame $\TF{acq}{ptu}$, which is dynamic and depends on the PTU position for each laser scan, and the transformation from the PTU frame and the laser scanner frame $\TF{ptu}{laser}$, which is static. This two transformations can be chained together to obtain the point in the acquisition frame, according to:


\begin{figure}
    
    \centering
    \begin{tikzpicture}
        % \node[draw, ellipse] (scene) {$scene$};
        % \node[draw, ellipse, right of=scene, xshift=3cm] (acquisition) {$acquisition$};
        % \node[draw, ellipse, right of=acquisition, xshift=3cm] (ptu) {$PTU$};
        % \node[draw, ellipse, right of=ptu, xshift=2cm] (laser) {$laser$};
        
        % \draw[-latex] (scene) -- (acquisition)
        % node[midway, above] {$\TF{scene}{acq}$};
        % \draw[-latex] (acquisition) -- (ptu)
        % node[midway, above] {$\TF{acq}{ptu}$};
        % \draw[-latex] (ptu) -- (laser)
        % node[midway, above] {$\TF{ptu}{laser}$};

        \tikzset{
            frame/.style={draw, ellipse, minimum width=3cm},
        }

        \node[frame] (scene) at (0,0) {scene};
        \node[frame] (acquisition) at (0, -2) {acquisition};
        \node[frame] (ptu) at (0, -4) {PTU};
        \node[frame] (camera) at (3, -6) {camera};
        \node[frame] (laser) at (-1, -8) {laser scanner};

        \draw[-latex] (scene) -- (acquisition)
            node[midway, right] {$\TF{scene}{acquisition}$};
        \draw[-latex] (acquisition) -- (ptu)
            node[midway, right] {$\TF{acquisition}{PTU}$};
        \draw[-latex] (ptu) -- (camera)
            node[midway, right] {$\TF{PTU}{camera}$};
        \draw[-latex, blue] (ptu) -- (laser)
            node[midway, below right] {$\TF{PTU}{laser}$};
        \draw[-latex, red] (camera) -- (laser)
            node[midway, right] {$\TF{camera}{laser}$};
    \end{tikzpicture}
    
    \caption{Transformation graph.}
    \label{figure:geometric-transformation-graph}
\end{figure}

\begin{equation}\label{eqn:point-registration}
    p_{i,j} = 
    \left[
        \begin{array}{c}
            x_{i,j} \\ y_{i,j} \\ z_{i,j} \\ 1 \\
        \end{array}    
    \right]
    =
    \TF{acq}{ptu}
    \cdot \TF{ptu}{laser}
    \cdot
    \left[
        \begin{array}{c}
            r_i \cos(\theta_i) \\
            r_i \sin(\theta_i) \\
            0 \\
            1 \\
        \end{array}    
    \right]
\end{equation}

At this phase, each point has 2 indexes, one for the laser scan index $j=1\dots L$ and another for the range index $i=1\dots N$, relative to the each laser scan. Therefore, at this stage, each point can be indexed with a pair of $(i,j)$ indexes. This point clouds are called structured point clouds.

This reconstruction phase depends heavily on the transformation from the PTU to the laser scanner. This transformation is obtained by a calibration process and is commonly referred to as the extrinsic calibration of the laser scanner.

In conclusion, for each acquisition results a point cloud with $L \times N$ points, where $L$ are the number of laser scans and $N$ the number of range values in each laser scan. Each point can be indexed in a bidimensional space, which is useful for subsequent algorithms.
