\chapter{Introduction}

Digital reconstruction of three dimensional scenes is a field that gained a high importance in areas like architecture, robotics, archeology and autonomous driving, and its goal is to produce high-detail and accurate models of real 3D scenes. These models can be then used, for example, in virtual reality, to provide and immersive experience, as if the used was in the real scene. Nowadays, 3D reconstruction is even available in, usually targeting Augmented Reality applications.

\section{Motivation}

3D reconstruction is still a challenge. First, real scenes have a complex geometric and objects can have small details that are hard to reproduce. Secondly, the measurements from sensors are subject to noise and errors, and are limited. For example, LiDAR lasers scanners are incapable to capture the color of the objects.

Past experience tells that a single sensor is not enough to model real environments, so currently the process lies into using multiple sensors and trying to merge the data from all the sensors, in order to capture a more realistic model.

However, this introduces a set of other problems inherent to this approach. One challenge is the registration of the different sensors, so that the data from one sensor can be accurately merged with the data from the other sensor. More specifically, the positions of the sensors need to be known accurately as well as their internal parameters. These parameters are determined using calibrations methods that need to be robust and precise.

Current reconstruction algorithms require a large amount of manual work, which means that a reconstruction require many man-hours to be completed, which is unfeasible for most applications. One of the goals of reconstruction is to develop algorithms that reduce human intervention, making it faster and more accessible. 

\section{Problem Description}

Lidar laser scanners are becoming more available and more accessible,  and because of their properties, as their high precision and high range, became an unmatched technology for 3D reconstruction. The LiDAR lasers are available as 2D laser scanners or 3D laser scanners, like the lasers from FARO and Riegl. Despite their immense potential, 3D laser scanners are still a very expensive solution and cheaper solutions are comprised of a cheaper 2D laser scanner mounted on a moving frame. This solution, despite its low cost, can achieve good results, but requires a fine calibration between the laser and the moving frame. 

Also, laser scanners do not register the color information, so a common practice is to pair the laser scanner with a camera to get both geometric and color data. This method also requires a fine calibration between both sensors to correctly merge the data. 

\section{Objectives}

The objective is to develop a fully integrated solution for 3D acquisition using both laser scanner and a camera. This objective was divided into four main objectives:

\begin{itemize}
    \item develop a 3D laser scanner, consisting of a laser scanner and a camera, and capable of recording data from both sensors in a fast and semi-autonomous way;
    \item define a methodology to record the data from the scene. In particular, it has to define the movements of the moving frame of the laser scanner and the camera and also when and where the laser scans and the images are recorded. This methodology should take into account the limitations of both sensors and the geometry of the scene;
    \item develop a set of methods to reconstruct the geometry of the scene reliably. The main challenge is the laser scanner to PTU extrinsic calibration, because it is fundamental for a reliable reconstruction;
    \item develop a set of methods to merge the image data with the geometry of the scene, to obtain a fully colorized 3D model.
\end{itemize}

The final result is, then, a point cloud with color and geometric information.

\section{Document Outline}

This dissertation is composed of eight chapters, which are arranged as follows:

\begin{description}
    \item[Introduction] The current chapter, in which the description of the problem is shown and the objectives of this work are defined.
    \item[State of Art] Describes the technologies and solutions found in the field of 3D reconstruction, both commercial and academic, as well of a small technical background on this solutions.
    \item[Experimental Infraestructure] Introduces all the software and hardware used to develop this work. In particular, the mobile robot for 3D scanner is described.
    \item[Methodology for Scene Capture] Describes the methods and algorithms used to reconstruct the geometry of the scene, using the laser scan data recorded.
    \item[Methodology for Image Reconstruction] Describes the methods and algorithms used to reconstruct the color information of the scene, using the camera images recorded.
    \item[Result and Discussion] Presents and discusses the experimental results obtained in this work.
    \item[Conclusion and Future Work] Summarizes the overall work developed and present possible future work.
\end{description}