
\TitlePage
\vspace*{55mm}
\TEXT{\textbf{Resumo}}
    { Reconstrução tridimensional é uma área com múltiplas aplicações em arquitetura e robótica. Inúmera tecnologias existem para este efeito, como por exemplo a estereoscopia e luz estruturada. Contudo, muitas tecnologias carecem de precisão geométrica, que é por vezes um requisito. Outra tecnologia - LiDAR - é usada por causa dos seus resultados geométricos inigualáveis. No entanto, LiDAR é incapaz de capturar a cor dos objetos e uma solução é a integração de uma câmara com o LiDAR.\\
      Assim, neste trabalho foram desenvolvidos um conjunto de técnicas e algoritmos direcionados para a reconstrução 3D com LiDAR e câmara. Além disso, um scanner 3D foi desenvolvido para registrar cenas reais. Em particular, um método de calibração inovador foi desenvolvido para a calibração do laser, com precisão superior a um método semelhante.
      Finalmente, os métodos foram testados com dados de cenas reais. A reconstrução geométrica foi bem sucedida mas o registo de cor ficou aquém do que era esperado, por causa de uma calibração pouco precisa da câmara.
    }
\EndTitlePage
\titlepage\ \endtitlepage

\TitlePage
\vspace*{55mm}
\TEXT{\textbf{Keywords}}
    {3D reconstruction, Laser scanner, Point Cloud, Calibration, Normal Estimation, Color Fusion, Color Registration}
\vspace*{15mm}
\TEXT
    {\textbf{Abstract}}
    { Tridimensional reconstruction is still a challenging area, that has multiple application in architecture and robotics. Several technologies are used today, like Stereoscopy or Structured Light, however, none is able to achieve precise geometric results, which are usually required. A technology, LiDAR, has evolved as the \textit{de facto} technology for tridimensional reconstruction, being able to achieve unmatched results. Yet, this technology is unable to register the color of objects, so the usual solution is the use a camera for this.\\
    Therefore, in this work we develop a set of algorithms and techniques for tridimensional reconstruction with a LiDAR laser scanner and a camera. Moreover, a 3D scanner was developed to register real-word scenes. In particular, a innovative calibration method was developed to calibrate the laser scanner, which performed above a similar calibration method.
    Finally, six reconstruction were done to test the algorithms developed. The geometric reconstruction was very accurate but the color reconstruction was perfect, specially because of the poor calibration method of the camera.}
\EndTitlePage
\titlepage\ \endtitlepage
